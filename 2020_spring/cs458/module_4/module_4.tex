\documentclass{article}

\usepackage{times}
\usepackage{textcomp}
\usepackage{listings}
\usepackage{fullpage}
\usepackage{color}
\usepackage{courier}
\usepackage{verbatim}
\usepackage{graphicx}
\usepackage{amsmath, amsfonts, amssymb, amsthm}
\usepackage{hyperref}
\graphicspath{{./}}

\lstset{language=c, keywordstyle={\bfseries \color{red}}, basicstyle=\footnotesize\ttfamily}
\setlength{\paperheight}{11in}
\author{Clement Tsang}

\begin{document}

\begin{center}
    \Large{CS 458 --- Module 4}
\end{center}

\section{Intro to Networks}
\begin{itemize}
    \item To create a network, you need 3 things:
        \begin{enumerate}
            \item Devices able to receive and send signals
            \item A way to connect devices to each other
            \item Rules for communicating, or a protocol
        \end{enumerate}
    \item Some examples of protocols:
        \begin{itemize}
            \item Token ring --- a person can only talk if they have the token
            \item CSMA/CD --- all listen to the wire, if they hear no signal they try to transmit.  If there is a collision, then they all stop and resend.
        \end{itemize}
    \item Well what are some problems?
        \begin{itemize}
            \item The Internet's design connects many computer networks together.  It also assumes that participants are honest and will cooperate --- they will not look at messages that don't belong to them, they will not delete your messages, etc.  Everyone should mutually work together\dots right?
            \item There's also no routing logic in the addressing scheme --- given some IP address, who knows where it comes from?  For example, a phone number has an area/country code.  An IPV4 address like \lstinline{136.192.63.0} could come from anywhere!
            \item Nor can you control the path your message follows!
            \item Your message can be broken up with each part following a different route.
            \item There is no real hard stop limit to the number of nodes (at least everywhere).
            \item It's really hard to conceptualize.
            \item Nobody is in charge (both good and bad).
        \end{itemize}
\end{itemize}

\section{Daemons, Servers, Ports}
\begin{itemize}
    \item A server is a computer on a network to do tasks for other computers (clients).
    \item A daemon is like a servant that can only do one task within a server.
    \item We can think of a server like a huge apartment building, and each apartment can have one servant (daemon).
    \item For example, the mail sending daemon (SMTP) is 25.
    \item Some apartments (ports) can be empty.  Many ports are actually empty!
    \item One could hide a service in a port it's not supposed to be in.
    \item For example, an HTTP daemon is in port 80.  This is implied by default (ie: \lstinline{https://www.uwaterloo.ca} implies \lstinline{https://www.uwaterloo.ca:80}).
    \item But one could put a web service at, for example, port 8080.
    \item A ``loose-lipped'' system may reply to an attacker and advertise what services they are running \emph{and} what at what port.
\end{itemize}

\end{document}
