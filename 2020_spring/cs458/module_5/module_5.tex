\documentclass{article}

\usepackage{times}
\usepackage{textcomp}
\usepackage{listings}
\usepackage{fullpage}
\usepackage{color}
\usepackage{courier}
\usepackage{verbatim}
\usepackage{graphicx}
\usepackage{amsmath, amsfonts, amssymb, amsthm}
\usepackage{hyperref}
\graphicspath{{./}}

\lstset{language=c, keywordstyle={\bfseries \color{red}}, basicstyle=\footnotesize\ttfamily}
\setlength{\paperheight}{11in}
\author{Clement Tsang}

\begin{document}

\begin{center}
    \Large{CS 458 --- Module 5: Internet Application Security and Privacy}
\end{center}

\section{Cryptography}
\begin{itemize}
    \item Recall Kerchhoffs' Principle/Shannon's Maxim --- \emph{assume the enemy knows the system}.
    \item We assume that Eve:
        \begin{itemize}
            \item May know the algorithm
            \item May know part of the plaintext
            \item May have many plaintext/ciphertext pairs
            \item May have access to an encryption/decryption oracle
        \end{itemize}
\end{itemize}

\subsection{Secret-key Encryption}
\begin{itemize}
    \item AKA symmetric encryption.
    \item Requires using a key to encrypt and decrypt the message.
    \item A theoretically perfect cryptosystem is the OTP; but this is very hard to use in practice since any key reuse ruins it, and must be truly random (pseudorandomness is not enough).
    \item SKES relies on mathematical problems that are hard to do in the opposite direction.  This may not hold true with quantum computing for some currently-used options!
    \item Two main types --- stream and block ciphers.
    \item A stream cipher takes in a keystream, XORs with the PT, and gets the CT.
    \item Can be very fast, and useful if you need to send a lot of data securely, but can be tricky to use correctly.
    \item To avoid key re-use, we may use IVs/salts to help make the key different.
    \item Block ciphers operate by taking part of the plaintext and spitting out the corresponding ciphertext.
    \item What happens when the PT is larger than one block?  We have \emph{modes of operation} for this:
        \begin{itemize}
            \item ECB uses the same keystream for each block --- but this degrades into a shitty substitution cipher!
            \item Other examples are CBC, CTR, GCM, etc, that starts with an IV. 
        \end{itemize}
    \item One problem with SKES is how to share keys?
\end{itemize}

\subsection{Public-key Encryption}
\begin{itemize}
    \item Allows Alice to send a secret message to Bob without a pre-arranged shared secret!
    \item Each user has a public key and a private key.
    \item These use a lot of math to allow for Alice to encrypt via Bob's public key and Bob to decrypt via his private key. 
    \item One downside with PKES is that a key size must be \emph{much} larger to get equivalent security to, for example, an AES SKES key.
    \item Another is that with larger keys, means more time.  This is slow to practically use for some use cases!
    \item We may instead use \emph{hybrid} cryptography.  This uses PKES to encrypt smaller messages, like a key for a SKES system, then encrypt the messages via SKES!
\end{itemize}


\end{document}
