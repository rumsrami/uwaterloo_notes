\documentclass{article}

\usepackage{times}
\usepackage{textcomp}
\usepackage{listings}
\usepackage{fullpage}
\usepackage{color}
\usepackage{courier}
\usepackage{verbatim}
\usepackage{graphicx}
\usepackage{amsmath, amsfonts, amssymb, amsthm}
\usepackage{hyperref}
\graphicspath{{./}}

\lstset{language=python, keywordstyle={\bfseries \color{green}}, basicstyle=\footnotesize\ttfamily}
\setlength{\paperheight}{11in}
\author{Clement Tsang}

\begin{document}

\begin{center}
    \Large{LS 101 --- Chapter 6: Civil Litigation and Tort Law} 
\end{center}

\section{Civil Litigation and the Trial Process}
\begin{itemize}
    \item Most civil cases don't make it to trial.
    \item Pre-trial procedures:
        \begin{itemize}
            \item Notice of action --- informs the defendant that an action has been started
            \item Statement of claim --- the plaintiff must make a claim that outlines their allegations and damages and includes supporting documents (process is AKA pleadings).
            \item Statement of defence --- the defendant prepares a statement of defence to counter-claim.
        \end{itemize}
    \item Discovery --- involves an examination of documents by the lawyers.  This may involve the testimony and questioning under oath of both parties by both lawyers.  The object is to make all evidence/allegations known and ensure there aren't surprises in court.  These are usually transcribed by a court reporter and part of the evidence at trial.
    \item In-chamber meetings --- lawyers may meet up with a judge who is not trying the case to explain why a settlement cannot be reached, and the judge will see if there are merits of the case, weaknesses, and pressure sides to come to an agreement.
    \item Trial process
        \begin{itemize}
            \item Both the plaintiff and defendant are testifying under oath and being questioned/cross-examined.
            \item Civil cases are heard by only a judge unless it's a personal injury case.
            \item Strict rules of evidence followed (ie: the lawyer cannot use leading questions but the one doing cross-examination can).
            \item Hearsay evidence is not admissible.
            \item The judge then makes a decision a few days/weeks later.
            \item The judge may award special damages (specific accountable costs), general damages (cannot be clearly specified), punitive damages (deterrence), lawyer's fees, a cease and desist, or specific obligations.
        \end{itemize}
    \item Small claims court hears cases that are below \$10 000 (though this is changing to be \$25 000 in Ontario).  Lawyers are usually not present since the amounts contested and the legal costs awarded are relatively small.  The judge takes a more active role questioning/cross-examining witnesses, and judgements are usually made on the spot.
    \item Appeals can be made based on questions of law, but not on questions of fact.  The costs of appeals are prohibitive since transcripts alone can be thousands of dollars.  The appellant launches the appeal and the respondent replies to the action; appeals can be disregarded.  If appeals are heard, the appeal court may uphold, order a new trial, or overturn the decision.
    \item The process of recovery requires that the victor get their money.  The cost of bringing an action may not be worth the reward if the defendant avoids payment.  The victor can take steps to garnishee money via wages/bank accounts, though only a portion of a debtor's wages can be seized as by law they must have enough to live.
\end{itemize}

\section{Tort Law}
\begin{itemize}
    \item Tort --- derived from Latin for ``crooked'' or ``wrong''.
    \item Tort law falls under civil law.
    \item When one party has caused harm to another party or their property.
    \item Used in the absence of contracts.
    \item A tort action is an attempt to seek compensation by using a civil law on a private action (ie: you sue someone for harming you, that lawsuit is a tort).
    \item Begins when the plaintiff sues the defendant/respondent.
    \item The difference between crime and torts is that a tort is based on a private incident.  A crime is an offence against the public and the state (ie: you see it as like State vs. Defendant).
    \item Some overlaps between torts and crime --- for example, assault is probably both a crime and a tort.
    \item Tort's are often not followed up though --- for example, you might not bother if the defendant clearly wouldn't be able to pay anyways.
    \item Torts require less proof --- you just need to show that the defendant's actions caused harm.
    \item Vicarious liability is when one is held responsible for the actions of someone else (your own employees, your soldiers, your partners, etc.).
    \item Joint liability refers to situations where several people are held jointly responsible for some harm committed (ie: business partners).
    \item If a defendant is found to be responsible for the plaintiff's injury, the victim can receive damages.  Types are special, general, or punitive (these are rare in Canada and are meant to punish/make an example out of).
    \item An injunction is when the court orders the defendant to cease their wrongful conduct.
    \item Torts can be classified as intentional and negligence.  Most civil court cases deal with negligence.
    \item For a negligence suit to succeed, 5 elements must be present:
        \begin{enumerate}
            \item Duty of care --- was there a duty of the defendant to be careful at such time?
            \item Did the defendant's behaviour fall below that standard (fault/breach)?  Was the harm reasonably foreseeable?
            \item Loss/damage/injury must be present for liability; there must be some sort of loss of injury.
            \item Is there causation between the damage/injury and the conduct of the defendant?
            \item There must be no prejudicial conduct (contributory negligence) on the part of the victim.
        \end{enumerate}
    \item Premise liability is when one has to require owners of property to ensure that nobody suffers injury/harm on that property within reason.  For example, stairs should have guardrails, wet floor signs should be placed if the floors are slippery, operating fire alarms/exits should exist, etc.
\end{itemize}

\subsection{Defences to Torts}
\begin{itemize}
    \item There are numerous possible defences to torts.
    \item For example, one can claim that it was an accident, and that the defendant had no ability to foresee the danger, had no control, or could not have avoided it even with the greatest of care/skill.
    \item Necessity refers to the self-preservation or the preservation of others --- for example, maybe they had to trespass to avoid a physical threat.
    \item Police have lawful authority to use force during an arrest.
    \item Self defence or defence of property can be used for charges of assault.
    \item Consent is a defence (ie: you can't sue for injury if it happened during a boxing match).
    \item Truth and fair comment (more for defamation).
    \item Victims may also be disqualified from seeking damages if they voluntarily assume the risk, though courts are reluctant to use this against a victim unless they also give up any claim to damages.
\end{itemize}

\subsection{The Law in Practice}
\begin{itemize}
    \item Insurance is a variable in lawsuits --- now, that might be a source of money that one could take!
    \item This is a bit of a debated issue; now the punishment kinda goes onto the insurance company.
    \item No-fault automobile insurance is an alternative to torts --- so if you get into an accident, damages to the vehicle are dealt with by your insurance company, no need to get payment from the other party.
    \item Worker's compensation can also be seen as an alternative to torts, rather than suing the workplace.
    \item Torts can be a major mechanism for social change and reform.
\end{itemize}

\subsection{Increase in Tort Actions}
\begin{itemize}
    \item There have been more tort actions.
    \item There may be various factors:
        \begin{itemize}
            \item Changing values make lawsuits more acceptable; they used to be more stigmatized.
            \item The duty of care has expanded.
            \item The number of lawyers has increased, and they might think they can win.
            \item Contingency fees are a deal between a lawyer or a client where the lawyer doesn't take (much) money directly, but they get a percentage of the winnings.  This also means the lawyer will want to convince to sue.
            \item Torts may be launched for other motives (spite, moral victory, publicity, discourage people, etc.).
            \item An entitlement mentality.
            \item Expanding laws.
        \end{itemize}
\end{itemize}

\subsection{Frivolous and Abusive Lawsuits}
\begin{itemize}
    \item Stella awards are based on the lawsuit where someone won for spilling coffee on themselves.
    \item These are for the most frivolous but successful lawsuits.
    \item Fly in the water bottle case.
    \item \$65 million dollar pair of pants being lost.
\end{itemize}

\end{document}
