\documentclass{article}

\usepackage{times}
\usepackage{textcomp}
\usepackage{listings}
\usepackage{fullpage}
\usepackage{color}
\usepackage{courier}
\usepackage{verbatim}
\usepackage{graphicx}
\usepackage{amsmath, amsfonts, amssymb, amsthm}
\usepackage{hyperref}
\graphicspath{{./}}

\lstset{language=python, keywordstyle={\bfseries \color{green}}, basicstyle=\footnotesize\ttfamily}
\setlength{\paperheight}{11in}
\author{Clement Tsang}

\begin{document}

\begin{center}
    \Large{LS 101 --- Chapter 5: Intro to Canada's Legal History, Court System, and Chater of Rights and Freedoms}
\end{center}

\section{Intro to Canada's Legal History}
\begin{itemize}
    \item British North America Act --- Britain could not modify laws regarding Canada without Canada's say
    \item Canada Act --- constitution upon which powers are based for Canada
    \item Both the federal and provincial government have three distinct branches --- legislative, executive, and judicial.
        \begin{itemize}
            \item Legislative branch is the most powerful and can make any laws as long as they are consistent with the constitution.
            \item The executive branch includes all government branches.  Their powers are determined by legislation and are empowered to run various government programs, implement and enforce various laws passed by the legislative branch, and often their powers also include the right to create rules/regulations with the force of law.
            \item The judicial branch of government includes the courts and judges.  They are expected to be independent/free with their decisions and should only be concerned with law, not politics.  Courts can also strike down laws.
        \end{itemize}
    \item Judges have been important throughout history with the creation of ``common law'', which evolved in Britain over the use of judges who would travel the countryside on circuit to hear and arbitrate cases.
    \item These judges did not have written laws and were often guided by local customs/traditions/values and their own judgement.
    \item As cases and laws evolved, attorneys to represent clients arose.
    \item Eventually, the practice of following precedents had become law, and so judges had essentially created a body of civil/criminal law that was common to people in England.
\end{itemize}

\section{Overview of the Court System in Canada}
\begin{itemize}
    \item Provincial courts:
        \begin{itemize}
            \item Lower/inferior courts.
            \item Usually several divisions (youth,  traffic, family, etc.).
            \item Judges paid and appointed by provinces.
            \item Small claims courts are an example.
            \item Usually deal with less serious summary conviction offences.
        \end{itemize}
    \item Superior courts:
        \begin{itemize}
            \item AKA Supreme Court, Superior Court,  Court of Queen's Bench, etc.
            \item Two levels --- trial and appeal court.
            \item More formal than lower level provincial courts; parties more likely to be represented by counsel.
            \item Judges in superior courts are appointed and paid by the federal government.
        \end{itemize}
    \item Appeal Courts
        \begin{itemize}
            \item Each province has an appeal court.
            \item Appellant must satisfy the court that the trial-level judge did not follow rules/procedure or made an error.
            \item Appeal court can overturn a verdict or order a new trial.
        \end{itemize}
    \item The Supreme Court of Canada
        \begin{itemize}
            \item Hears appeals from the courts of appeal of all provinces.
            \item Usually only when the issue has significance to the law in Canada.
            \item Because the Charter gives the Supreme Court a significant role in interpreting the law, the appointment of Supreme Court Justices now has some political significance.
        \end{itemize}
    \item The court system is meant to ensure proceedings are fair, impartial, and legitimate.  One way to do so is to be open.
    \item On occasion, though, witnesses may be excluded from part of a hearing so their testimony isn't altered by listening to others.
    \item Or, judges often place media bans on publishing the names of sexual assault victims and children, evidence in preliminary hearings, and names of young people involved.
    \item The Criminal Justice Act allows a judge to exclude any person from a hearing to protect child witnesses and in the interest of public morals.
\end{itemize}

\section{The Canadian Charter of Rights and Freedoms}
\begin{itemize}
    \item Supreme law of Canada, after the Canada Act.
    \item Gives the government of Canada the sole responsibility of managing our law.
    \item The Charter does not regulate private citizens directly, but it does indirectly affect them.
    \item Also, each province has their own human rights code, but they must all be in accordance with the Charter.
    \item Note that there are some limitations on the Charter, if the government can justify them in a free and democratic society.
    \item One example is if it is for the greater good for society.
    \item Or if it protects other people's rights.  For example, there are reasonable limits on the freedom of speech.
    \item The scope of the Charter is not absolute.
    \item There is a ``notwithstanding clause'' in the Charter, which allows a provincial/federal government to pass a law that violates fundamental rights by invoking this clause.  Note that there is the ``sunset'' clause that voids the legislation after five years.
    \item Sections 7 to 14 cover the legal rights that are protected.
    \item Sections 7 to 10 deal with the right to life, liberty, and security of the person.
        \begin{itemize}
            \item Abortions are legal by our constitution (does not extend to an unborn fetus).
            \item There are rights to be secure from unreasonable search/seizure.
            \item The police do have the rights to search suspects who are under arrest, though.
            \item There is the right to not be arbitrarily detained/imprisoned, though the police do, again, have extensive rights to detain suspects.
            \item The right to consult and contact a lawyer.
            \item The right of habeus corpus (right to go to court).
        \end{itemize}
    \item Section 11 deals with criminal proceedings and trials.  The accused has the right:
        \begin{enumerate}
            \item Be informed without reasonable delay of the specific offence
            \item To be tried within a reasonable time
            \item Not to be compelled to be a witness against themselves
            \item Presumed innocent until proven guilty
            \item Not to be denied reasonable bail without just cause
            \item To be tried by jury if maximum penalty is imprisonment over 5 years
            \item Not to be found guilty for an offence that was not an offence when committed
            \item If found guilty/not guilty of an offence to not be tried again for the same offence
            \item If found guilty, and the punishment has been changed since the commission of the offence, be afforded the lesser punishment
        \end{enumerate}
    \item Sections 12 to 14 prohibits cruel and unusual punishment; witnesses may testify without their testimony being used against them, and parties in any proceedings have the right to the assistance of an interpreter.
    \item Section 15 deals with equality rights --- discrimination by race/nationality/ethnic origin/religion/sex/disabilities us prohibited, and individuals have the right to equal protection and benefit of the law.  This section was used to legalize gay marriage.
    \item The Charter effectively gives the courts a veto over legislation.
    \item The courts have the power to strike down or amend existing legislation.
\end{itemize}

\end{document}
