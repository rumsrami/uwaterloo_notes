\documentclass{article}

\usepackage{times}
\usepackage{textcomp}
\usepackage{listings}
\usepackage{fullpage}
\usepackage{color}
\usepackage{courier}
\usepackage{verbatim}
\usepackage{graphicx}
\usepackage{amsmath, amsfonts, amssymb, amsthm}
\usepackage{hyperref}
\graphicspath{{./}}

\lstset{language=python, keywordstyle={\bfseries \color{green}}, basicstyle=\footnotesize\ttfamily}
\setlength{\paperheight}{11in}
\author{Clement Tsang}

\begin{document}

\begin{center}
    \Large{LS 101 --- Chapter 7: Contracts and Contract Law} 
\end{center}

\section{Fundamental Principals of Contracts}
\begin{itemize}
    \item Contracts are supposed to protect both sides.
    \item A contract is defined as an agreement between two or more parties that is binding in law.
    \item Usually the punishment for breaching the contract is financial damages.
    \item A bilateral contract is when both parties make a promise (I agree to sell you my house and I expect X amount from you).
    \item A unilateral contract is when only one party makes a promise (ie: providing information for an arrest).
    \item Each person should be giving something and getting something in return.
    \item Under a contract, you must do what you agreed to do.
    \item The heart of the contract is a consensus/agreement between the parties.
    \item Technically doesn't even need to be in written form, though this is better for obvious reasons --- it is much harder to prove if the other side disputes an oral contract.
    \item Some contracts may be required to be in writing (ie: contracts that are done over a long period of time, interest, land) in some places.
    \item Written contracts will be very hard to dispute, and is very strong evidence that people agreed to something.
    \item A contract between a debtor and paying a debt may be protected by an indemnity.  So if you tried to renege on it, there would be consequences as stipulated by the contract and they may have to pay compensation as per the indemnity.
    \item Courts are generally not interested whether a contract is fair/appropriate --- the courts will mainly be interested in enforcing whether the contract is being carried out.  However, this is changing a bit for some contracts (ie: consumer protection), and some courts may interfere.
    \item Many contracts don't allow us to negotiate the terms/conditions.
\end{itemize}

\section{Remedies to Breaches of Contract}
\begin{itemize}
    \item \emph{Quantum meruit} is ``a reasonable amount'' for what you should pay for breaching a contract (as monetary damages).
    \item This is meant to protect workers when clients refuse to pay an amount.
    \item Damages may be adjusted if the plaintiff hasn't taken steps to minimize the loss.
    \item An injunction is used to make a person to stop doing something when they are doing something inconsistent to the contract.
    \item The last way to get something when a contract is breached is ``specific performance'' --- the court orders the defendant to do something that they had agreed to in the contract, and can be held in contempt if they do not comply.
    \item Obviously there are some circumstances where you can get out of a contract as decided by a court --- perhaps the circumstances make it impossible/bring about undue hardships.
\end{itemize}

\section{Ingredients in a Contract}
\begin{itemize}
    \item Three parts to a contract:
        \begin{enumerate}
            \item Agreement/consensus
            \item Both parties must provide consideration
            \item Intend to be legally bound
        \end{enumerate}
\end{itemize}

\subsection{Agreement or Consensus}
\begin{itemize}
    \item The foundation of the contract.
    \item The courts have to judge whether the parties understood to what they agreed to and agreed to the same thing.
    \item The courts also expect you to be a reasonable person, and if its reasonable to have understood one's obligations for signing the contract, then one cannot claim to have misunderstood.
    \item A contract requires an offer and an acceptance.  An offer for a reward, for example.
    \item An agreement expresses that the party is willing to be bound legally to those terms.  These agreements can be verbal/written.
    \item Again, courts will see if a reasonable person can have seen that an agreement was made.
    \item An advertisement is kinda like a contract, and are considered ``invitations'' --- they are not legal contracts, but they are somewhat similar (give an offer).
    \item An offer can be revoked, but must be communicated to the other party if it is being withdrawn, or if it is being rejected by another party.
    \item An offer that is not explicitly defined to be open for a specific period of time will be open for a reasonable amount of time.
    \item Acceptance of a contract must be communicated to the person who made the offer, and unqualified.  If one wants new terms in the offer, then they have done a counter-offer, not an acceptance.
    \item A counter-offer will void the original offer.
\end{itemize}

\subsection{Consideration}
\begin{itemize}
    \item A contract requires that both parties pay a price and receive a benefit.
    \item Consideration refers to anything of value, including goods/services.
    \item For example, when you get something repaired, the consumer is supposed to pay for the service.
    \item A gratuitous promise (a gift) is unenforceable --- only one end of the party is doing something, and so there are no obligations for either side.
    \item The consideration must be specific.  You can't say ``some money'', it has to be a hard value.
    \item The consideration must have some economic value.  For example, you can't make a contract for being loving or something.
    \item A consideration doesn't have to be adequate --- if you agree to it, that's on you.
    \item Now, if there's some circumstances causing the consideration to be very inadequate to one side (ie: fraud) and thus obviously unbalanced, the courts may step in.
    \item A contract is unenforceable if it is illegal or against public policy.
    \item You do not have to pay either if it was part of a person's public duty anyway to do something.  So, if you ask a police officer to remove someone who's misbehaving from an establishment for money, they are not able to collect if you refused to pay, since it was their job in the first place (dick move though).
    \item Considerations must be possible, as well.  You can't create a contract where a psychic brings back someone from the dead.
    \item A consideration cannot be collectable if it was in the past.  For example, you cannot collect if you, say, cut the grass for someone out of good will, then changed your mind and tried to collect.
\end{itemize}

\subsection{Intention and Consent}
\begin{itemize}
    \item Both parties must intend legal consequences to follow from their agreement.
    \item Contracts are usually about commercial relationships.
    \item Both parties must have the capacity to enter into an agreement.
    \item Minors typically do not have the legal capacity to enter a contract.
    \item Contracts may be voided if consent was obtained by undue influence (not true consent).
\end{itemize}

\section{Terms of a Contract}
\begin{itemize}
    \item The terms of a contract include the statements, promises, and stipulations.
    \item Local/business customs may be used to determine rights/obligations in cases where laws are insufficient.
    \item A term is an undertaking in the contract.
    \item A condition refers to an important term in the contract.
    \item A warranty denotes a less important term in a contract.
    \item Representations are statements that lead up to the contract, but not part of the contract itself.
    \item If a representation is false then one could seek damages or that the contract is nullified.
    \item Misrepresentations are not likely to void a contract unless they are substantial.
    \item Again, use a reasonable person test.  If a reasonable person wouldn't have believed the claims it will likely not be overruled.
    \item For example, you wouldn't expect buying a car to make you a babe magnet.
    \item Also, the seller doesn't have to tell a buyer if they are understanding something wrong --- this is \emph{caveat emptor}, or ``buyer beware''.  The buyer making a mistake in the contract doesn't void the contract!
    \item The exclusion clause aims to exclude or limit financial liability for one of the parties.  Historically courts aimed to restrict this, and would strike these down.  This was mainly because these aren't very equal exclusion clauses, and customers have no choice to sign!
\end{itemize}

\section{Discharging a Contract}
\begin{itemize}
    \item A contract can be discharged through performance --- both people do what their contract outlines and it ends.
    \item A contract can be discharged or mended by agreement of both parties.
    \item Frustration can refer to an outside event that prevents the fulfilment of the contract --- this could be law changes, acts outside of one's control, or radical changes of circumstances that require changing the contract.
    \item Obviously this is under reason; you can't burn down a house to avoid painting it.
    \item A contract is breached via incomplete/incompetent performance; a party can end the contract and sue for breach of contract.
    \item A contract is breached if one party refuses to perform/a major term has been breached.
\end{itemize}

\section{Basic Principles Underlying Criminal and Civil Law}
\begin{itemize}
    \item The Criminal Code is a federal statute.
    \item Criminal offences are crimes against the state.
    \item The criminal law aims to punish wrongdoers.
    \item Criminal acts carry social stigma.
    \item Criminal proceedings are meant to be fair and impartial.
    \item Criminal offences must be clear and on the books.
    \item A crime includes an \emph{actus reus} and a \emph{mens rea}, or ``guilty act'' and ``guilty mind'', or that the defendant did the guilty act with a guilty intention.
    \item The Crown Attorney must prove guilt beyond a reasonable doubt.
    \item Victims have little involvement in a criminal proceeding.
    \item Meanwhile, civil law deals with disputes between individual parties.
    \item The goal is to redress personal injuries.
    \item Usually tort and contract laws they are provincial.
    \item Usually less moral.
    \item Negligence is a form of intent found in both criminal and civil matters.
\end{itemize}

\end{document}
