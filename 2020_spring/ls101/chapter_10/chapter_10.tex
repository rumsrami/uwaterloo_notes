\documentclass{article}

\usepackage{times}
\usepackage{textcomp}
\usepackage{listings}
\usepackage{fullpage}
\usepackage{color}
\usepackage{courier}
\usepackage{verbatim}
\usepackage{graphicx}
\usepackage{amsmath, amsfonts, amssymb, amsthm}
\usepackage{hyperref}
\graphicspath{{./}}

\lstset{language=python, keywordstyle={\bfseries \color{green}}, basicstyle=\footnotesize\ttfamily}
\setlength{\paperheight}{11in}
\author{Clement Tsang}

\begin{document}

\begin{center}
    \Large{LS 101 --- Chapter 10: Family Law and Social Policy --- Marriage and Divorce}
\end{center}

\section{Sacred, Social, and Personal Concepts of Marriage}
\begin{itemize}
    \item The sacred concept sees marriage as a religious holy union.
    \item The contract is a contract the couple makes with some higher order.
    \item The authority resides with, for example, in Christianity, God and the Church.
    \item The social concept gives priority to family obligations, and emphasizes self-sacrifice.
    \item The authority here resides with kinship, parents, and elders.
    \item In the former two, divorce is stigmatized.
    \item The personal concept gives priority to individual rights and personal happiness.
    \item Laws regulating marriage and the family are largely a provincial responsibility.  Divorce laws are federal statutes.
\end{itemize}

\section{Defining Marriage}
\begin{itemize}
    \item The Ontario \emph{Family Law Reform Act} moved the province towards a limited legal recognition of common-law marriages.
        \begin{itemize}
            \item The Act defines a spouse for the purposes of support obligations as either a married person or a couple living together for 5 years continously, or within a relationship of some permeance where there is a child born and they have cohabited within the preceding year.
        \end{itemize}
    \item Bill C-38 legalized same-sex marriages.
    \item Marriage has 5 main characteristics --- it:
        \begin{enumerate}
            \item Is a socially legitimate sexual union
            \item Is a public affair that must be publically registered
            \item Is undertaken with some idea of permanence
            \item Involves emotional commitment and support
            \item Is a legal contract with specific rights and obligations
        \end{enumerate}
    \item Marriage has to be solemnized by a Judge, Justice of Peace, or someone authorized by the provincial government (ie: clergy, church).
\end{itemize}

\section{Legal Requirements to Marry}
\begin{itemize}
    \item To marry, each person must have the legal capcity to do so.
    \item If consent is coerced or affected by alcohol or drugs, the marriage can be declared nvoid and an annulment granted.
    \item Most places prohibit marriages below a certain age and between close blood relatives.
    \item Bigamy is prohibited in most jurisdictions.
    \item Ontario's \emph{Family Law Reform Act}s of 1978 and 1986 made changes to the legal rights and obligations of marital partners --- spouses have identical rights and obligations in marriage.
    \item The spouses should manage the moral and material direction of the family, exercise parental authority, and assume the resulting takss.
    \item Ontario has abolshed the distinction between legitmate and illegitmate children.
    \item In all provinces, marriage creates a statutory right to share in the value of property acquired during marriage.
    \item ``Net family property'' under the \emph{Family Law Act} should be divided equally between spouses unless it would be unfair.
\end{itemize}

\section{Divorce Legislation}
\begin{itemize}
    \item By law, people must divorce in order to remarry.
    \item No-fault divorce laws shift the focus of the legal process from the moral euqstions of fault and blame, to economic issues of marital property, custody, and support.
    \item The \emph{Divorce Act} of 1968 made divorce easier to obtain in Canada, and contains various grouds for divorce, including provisons for no-fault fdivorce.
    \item Section 3 allows for divorce on various sexual grounds, and physical/mental cruelty.
    \item Section 4 allows divorce when one spouse is imprisoned, addicted to drugs without prospect of rehabilitation, failure to consummate a marriage, and mental breakdown.
    \item A divorce can also be granted if a spouse can prove desertation (3 years apart) or if 5 years with no requirement to prove a ``fault''.
    \item In 1985, the \emph{Divorce Act} further liberalized divorce laws, which allows couples to obtain a divorce on the grounds that there was a breakdown of their marriage as per section 8.
    \item A breakdown of a marriage is established if:
        \begin{itemize}
            \item The spouses have lived apart for at least 1 year immediately preceeding the determination of the divorce proceedings or were living separate and apart at the commencement of the proceedings; OR
            \item The spouse against wohm the divorce proceeding is brought has, since the marriage:
                \begin{itemize}
                    \item Committed adultery
                    \item Treated the other spouse with physical/mental cruelty
                \end{itemize}
        \end{itemize}
\end{itemize}

\section{Property}
\begin{itemize}
    \item Prior to 1978, only wives could obtian alimony and had to apply through the \emph{Deserted Wives' and Children's Maintenance Act}.  It would only be provided in cases where there was desertion by her husband and a failure to maintain her, the parties had a valid marriage, they were living apart, and the husband was at fault in some way (adultery, cruelty, desertion).
    \item The husband could get out of alimony if he could prove adultery on the part of his wife.
    \item Ontario now has a gender-neutral spousal support law, which can be found in the \emph{Family Law Act} of 1990.
    \item Usually, wives obtian custody of children in 85-90\% of cases, and courts will typically order fathers to pay child support.
    \item The amount of support awarded will depend on things like finanicial needs, their capacity to become financially independent, length of time together, accustomed standard of living, etc.
    \item Note a spouse's misconduct is not a factor in awarding support.
    \item Both the \emph{Family Law Act} and the \emph{Divorce Act} both place the primary obligation on spouses to support themselves to the extent that this is possible and practical.
\end{itemize}

\section{Marriage Contracts and Separation Agreements}
\begin{itemize}
    \item Prior to 1978's Ontario \emph{Family Law Reform Act}, domestic contracts were limited in law as they were thought to undermine the stability of marriage.
    \item Now, couples that marry or cohabit are allowed to construct a marriage/cohabitation agreement, which can include a separation agreement.
    \item These contracts must be in writing, signed by the parties, and witnessed.
    \item These contracts usually protect property that each person has brought into the marriage, divide property acquired during hte marriage, limit support obligations, for other person's children, etc.
    \item Note that the terms of a marriage contract/separation agreement are not necessarily binding on court, but carry weight.
    \item They can be disputed on grounds like fraud, duress, misrepresentation, inprovidence (manifest unfairness), coercion, and non-disclosure of assets.
    \item Courts have ruled that both parties of a domestic contract must enter it in good faith.
\end{itemize}

\end{document}
