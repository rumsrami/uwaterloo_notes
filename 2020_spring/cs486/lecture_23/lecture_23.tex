\documentclass{article}

\usepackage{times}
\usepackage{textcomp}
\usepackage{listings}
\usepackage{fullpage}
\usepackage{color}
\usepackage{courier}
\usepackage{verbatim}
\usepackage{graphicx}
\usepackage{amsmath, amsfonts, amssymb, amsthm}
\usepackage{hyperref}
\graphicspath{{./}}

\lstset{language=python, keywordstyle={\bfseries \color{blue}}, basicstyle=\footnotesize\ttfamily}
\setlength{\paperheight}{11in}
\author{Clement Tsang}

\begin{document}

\begin{center}
    \Large{CS 486 --- Lecture 23: Game Theory}
\end{center}

\section{Pareto Dominance}
\begin{itemize}
    \item An outcome $o$ Pareto dominates another outcome $o'$ iff every player is weakly better off in $o$ and at least one player is strictly better off in $o$.  That is:
        \begin{align*}
            U_i(o) \geq U_i(o'), \forall i \\
            U_i(0) > U_i(o'), \exists i
        \end{align*}
    \item A Pareto optimal outcome is an outcome $o$ such that no other outcome $o'$ Pareto dominates $o$.
\end{itemize}

\section{Prisoner's Dilemma}
\begin{itemize}
    \item Follows a game like so:
        \begin{table}[h!]
            \centering
            \begin{tabular}{|c|c|c|}
            \hline
                      & Cooperate & Defect   \\ \hline
            Cooperate & (-1, -1)  & (-3, 0)  \\ \hline
            Defect    & (0, -3)   & (-2, -2) \\ \hline
            \end{tabular}
        \end{table}
        So, the optimal move for \emph{all} of them is for them to both not say anything.
    \item But if they both snitch, they get the objectively \emph{worst} reward for all of them.
    \item And if only  one betrays the other, then one gets the worst possible reward for the player specifically, while one gets the best possible reward.
    \item By dominant strategy equilibrium, both players defecting is the dominant strategy!
    \item By pure-strategy Nash equilibria, we have 1 pure outcome --- both defecting!
    \item Now, let's apply Pareto optimality.  There are 3 Pareto optimal outcomes --- both cooperating, and one defecting (for both players).
\end{itemize}

\section{Mixed Strategy Nash Equilibrium}
\begin{itemize}
    \item Now let's consider a game like:
        \begin{table}[h!]
            \centering
            \begin{tabular}{|c|c|c|}
            \hline
                  & heads  & tails  \\ \hline
            heads & (1, 0) & (0, 1) \\ \hline
            tails & (0, 1) & (1, 0) \\ \hline
            \end{tabular}
        \end{table}
    \item This game does NOT have a pure-strategy Nash equilibrium!  Note this doesn't disprove the claim from before; all finite games have at least one \emph{mixed}-strategy Nash equilibrium.
    \item Let's assume Bob plays heads with probability $q$, and Alice with probability $p$.
    \item Alice should choose a value for $p$ such that Bob is indifferent between his actions --- in this case, $p = 0.5$.
    \item Likewise, $q = 0.5$ for the same reasoning, but now it's for Bob.
    \item Each player should, in general, choose their mixing probability to make other players indifferent between their actions.
\end{itemize}

\end{document}
