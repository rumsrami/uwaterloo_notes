\documentclass{article}

\usepackage{times}
\usepackage{textcomp}
\usepackage{listings}
\usepackage{fullpage}
\usepackage{color}
\usepackage{courier}
\usepackage{verbatim}
\usepackage{graphicx}
\usepackage{amsmath, amsfonts, amssymb, amsthm}
\usepackage{hyperref}
\graphicspath{{./}}

\lstset{language=python, keywordstyle={\bfseries \color{blue}}, basicstyle=\footnotesize\ttfamily}
\setlength{\paperheight}{11in}
\author{Clement Tsang}

\begin{document}

\begin{center}
    \Large{CS 486 --- Lecture 22: Game Theory}
\end{center}

\section{Intro to Game Theory}
\begin{itemize}
    \item Game theory: given a game, how would agents play it?
    \item Mechanism design: how should we design the rules of the game so agents behave the way we want them to?  Almost like reverse game theory.
    \item We focus mostly on game theory.
    \item A game can be:
        \begin{itemize}
            \item Cooperative where agents have a common goal.
            \item Competitive where agents have conflicting goals.
            \item Or something in between.
        \end{itemize}
\end{itemize}

\section{Dominant Strategy Equilibrium}
\begin{itemize}
    \item Consider a game, where we have two players, Alice and Bob (Alice is rows, Bob is columns):
        \begin{table}[h!]
            \centering
            \begin{tabular}{|c|c|c|}
            \hline
                & home   & dancing \\ \hline
        home    & (0, 0) & (0, 1)  \\ \hline
        dancing & (1, 0) & (2, 2)  \\ \hline
            \end{tabular}
        \end{table}
    \item Suppose both cannot communicate with each other.  So, they must independently make a decision.
    \item Furthermore, they must choose at the same time, and cannot observe the other player's action (simultaneous move game).
    \item We see that the utility is maximized by both of them dancing.
    \item Let us denote $\sigma_i$ as the strategy of player $i$, and $\sigma_{-i}$ as the strategies of all players except $i$.
    \item Let us denote $U_i(\sigma) = U_i(\sigma_i, \sigma_{-i})$ as the utility of agent $i$ under the strategy profile $\sigma$.
    \item We define a strategy, $\sigma_i$, to dominate another one, $\sigma_i'$, if:
        \begin{align*}
            U_i(\sigma_i, \sigma_{-i}) &\geq U_i(\sigma_i', \sigma_{-i}), \forall \sigma_{-i}\\
            U_i(\sigma_i, \sigma_{-i}) &> U_i(\sigma_i', \sigma_{-i}), \exists \sigma_{-i}
        \end{align*}
        Or in other words:
        \begin{itemize}
            \item The strategy is as good or better than the other strategy for all opposing strategies.
            \item The strategy must be strictly better for at least one opposing strategy.
        \end{itemize}
    \item A dominant strategy dominates \emph{all} other strategies.
    \item When each player has a dominant strategy, this is called a dominant strategy equilibrium.
\end{itemize}

\newpage
\section{Nash Equilibrium}
\begin{itemize}
    \item Now let's consider another game:
        \begin{table}[h!]
            \centering
            \begin{tabular}{|c|c|c|}
            \hline
                    & dancing & running \\ \hline
            dancing & (2, 2)  & (0, 0)  \\ \hline
            running & (0, 0)  & (1, 1)  \\ \hline
            \end{tabular}
        \end{table}
    \item So, we know the optimal move is for them to both dance.
    \item But it's also reasonable for them to both choose to go running.
    \item This game has \emph{no} dominant strategy equilibrium!  Bob's choice would make Alice prefer another action (game is symmetric so WLOG)!  So, there is no dominant strategy.
    \item So what can we do?
    \item The Nash equilibrium is our answer.
    \item Given a strategy profile, $(\sigma_i, \sigma_{-i})$, agent $i$'s strategy is a best response to other agents' strategies iff:
        \[
            U_i(\sigma_i, \sigma_{-1}) \geq U_i(\sigma_i', \sigma_{-i}), \forall \sigma_i' \neq \sigma_i
        \]
        In other words, given what other agents are doing, $\sigma_i$ is the best choice for \emph{me}.
    \item Nash equilibrium is a strategy profile $\sigma$ iff each agent $i$'s strategy $\sigma_i$ is a best response to the other agents' strategies $\sigma_{-i}$.
    \item Both (dancing, dancing) and (running, running) are Nash equilibria in the given game.
\end{itemize}

\end{document}
