\documentclass{article}

\usepackage{times}
\usepackage{textcomp}
\usepackage{listings}
\usepackage{fullpage}
\usepackage{color}
\usepackage{courier}
\usepackage{verbatim}
\usepackage{graphicx}
\usepackage{amsmath, amsfonts, amssymb, amsthm}
\usepackage{hyperref}
\graphicspath{{./}}

\lstset{language=python, keywordstyle={\bfseries \color{blue}}, basicstyle=\footnotesize\ttfamily}
\setlength{\paperheight}{11in}
\author{Clement Tsang}

\begin{document}

\begin{center}
    \Large{CS 486 --- Lecture 17: Decision Networks}
\end{center}

\section{The Weather DN}
\begin{itemize}
    \item Suppose we want to ask --- do we take our umbrella or not?
    \item Depending on the weather, we may bring our umbrella.
    \item But we don't actually see the weather directly, we see the \emph{forecast}.
    \item We can draw it like this:
        \begin{lstlisting}
        (Weather) ->  <Utility>
            |             A
            V             |
        (Forecast) -> [Umbrella]
        \end{lstlisting}
    \item A policy specifies what the agent should do under all contingencies.
    \item For each decision variable, a policy specifies a value for the decision variable for each assignment of values to its parents.
    \item So, for our weather decision network, how many policies are there?
    \item Forecast has 3 possible values, and for each value of forecast, there are 2 possible decisions.  This gives $2^3 = 8$ possible policies.
    \item We have two approaches to solving the network problem:
        \begin{itemize}
            \item Brute force: solve every policy's expected utility, and choose the best one.
            \item VEA.
        \end{itemize}
    \item Obviously we prefer VEA since brute force can result in having to calculate a very large number of policies compared to the number of nodes!
    \item Consider the policy $\pi_1$:
        \begin{itemize}
            \item Take the umbrella if the forecast is cloudy
            \item Leave the umbrella at home otherwise
        \end{itemize}
    What is the expected utility of this policy?
    \item In this case, using a brute force approach:
        \begin{align*}
            EU(\pi_1) &= P(norain) * P(sunny|norain) * u(norain, leaveit)\\
                      &+ P(norain) * P(cloudy|norain) * u(norain, takeit)\\
                      &+ P(norain) * P(rainy|norain) * (norain, leaveit) \\
                      &+ P(rain) * P(sunny|rain) * u(rain, leaveit) \\
                      &+ P(rain) * P(cloudy|rain) * u(rain, takeit) \\
                      &+ P(rain) * P(rainy|rain) * u(rain, leaveit) \\
                      &= 0.7 * 0.7 * 100 + \dots + 0.l3 * 0.6 * 0 \\
                      &= 64.05
        \end{align*}
    \item We see that this could take too long.  We'll now investigate VEA.
\end{itemize}

\subsection{VEA}
\begin{itemize}
    \item The general algorithm is as follows:
        \begin{enumerate}
            \item Remove all variables that are not ancestors of the utility node.
            \item Create factors.
            \item While there are decision nodes remaining:
                \begin{enumerate}
                    \item Sum out each RV that is not a parent of a decision node
                    \item Find the optimal policy for the last decision
                \end{enumerate}
            \item Return the optimal policies.
            \item Sum out all the remaining RVs to get the expected utility of the optimal policy.
        \end{enumerate}
    \item Let's apply this to our weather example:
        \begin{enumerate}
            \item Nothing to be done.
            \item Define three factors, f1(Weather), f2(Forecast, Weather), f3(Weather, Umbrella)
            \item Now:
                \begin{enumerate}
                    \item Weather is not a parent of any decision node.  Sum out Weather.  For example, f1 * f2 * f3 = f4(Weather, Forecast, Umbrella), then sum out f4 to get f5(Forecast, Umbrella).
                    \item Now let's find the optimal policy for, say, Umbrella.
                \end{enumerate}
        \end{enumerate}
\end{itemize}

\end{document}
