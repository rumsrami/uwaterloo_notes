\documentclass{article}
\usepackage{times}
\usepackage{textcomp}
\usepackage{listings}
\usepackage{graphicx}
\graphicspath{{./images/}}

\author{Clement Tsang}

\begin{document}

\section{Friends}
\begin{itemize}
\item A friend function of a class is defined out of the class' scope, but the function has direct access to all of the private fields of the class.
\item We use the ``friend'' keyword to do so.
\item Consider the following code:
\begin{lstlisting}
class Vec {
    int x, y;
public:
    friend ostream &operator<<(ostream &out, const Vec &v); 
};

ostream &operator<<(ostream &out, const Vec &v) {
    return out << v.x << " " <<< v.y << endl;
}
\end{lstlisting}
\item We can use this on a class as well, to allow a class to be able to access all fields of another class. 
\item For example, putting ``friend class ClassTwo'' in ClassOne will allow ClassTwo to be able to directly access all fields in ClassOne.
\item Note that friends are not mutual, and friendship is not inherited by children.  That is, if ClassThree is a child of ClassOne, ClassTwo cannot access ClassThree despite being able to access ClassOne.
\end{itemize}

\section{Unifed Modelling Language (UML)}
\begin{itemize}
\item A way to visualize the design of a system.
\item We can use this for C++, Java, etc.
\item An example would be for a Vector class.  A class in UML consists of the name, fields, then methods.
\item Use ``+'' to symbolize a public field, and ``-'' to symbolize a private field.
\item What we can use UML for is to show the RELATIONSHIP between classes and the like.
\item For example, we can create a ``Basis'' class that takes in two vectors.
\item Side note, we need to create ctors using MIL(member initialization base) for Basis as our Vec class doesn't cover a zero argument default ctor.
\item Remember, when we initialize an object, the following run in order:
\begin{itemize}
\item Alloc memory
\item Initialize/construct fields
\item Run ctor body
\end{itemize}
\item Back on topic:
\item We can see the problem:
\begin{lstlisting}
class Vec {
    int x, y;
    public:
        Vec (int x, int y): x{x}, y{y} {}
};

class Basis {
    Vec v1, v2;
    //Right now, we don't have default ctors for the Vec class!
    //Instead, we'll make a new Basis ctor.
    //We CANNOT initalize the fields in the ctor body, so we MUST use MIL.
    Basis():
        v1{0, 0},
        v2{0, 0}
    {}
};

Basis b; //This is now valid.
\end{lstlisting}
\item Basically, it's kinda like the charts you get if you look at class hierarchy on Netbeans.
\end{itemize}

\section{Composition}
\begin{itemize}
\item Our ``Basis'' class depends and uses a Vector class.  It is composed with it --- it relies on another class to exist.
\item Note the opposite is not true, as we can have Vectors without ever declaring a Basis.
\item Note that the Vector has no identity outisde of A.  That is, we cannot access the v1 or v2 fields without going through Basis first. Destroying a Basis object destroys the Vectors that are associated with it.
\item This extends to copying --- if a Basis is copied, then the Vectors are also copied.
\item For composition, we show this with a class, a black diamond, and linking to the class it is composed of.
\item [Basis]<//>----[Vec] represents Basis owning two Vectors.
\end{itemize}

\section{Aggregation}
\begin{itemize}
\item We represent this with A <>---- B (a white triangle).
\item It means that B exists within A, but apart from it.  That is, if A is destroyed, B still exists.  
\item If A is copied, B is not copied.
\item Essentially, B does not know about the existence of A.  Furthermore, B could belong to multiple A instances.
\item We do so by making B out of scope of A, and using pointers to B from A.  We also have to make sure that in our destructor for A, we do not destroy B (a shallow destructor).
\item A simple example of aggregation:
\begin{lstlisting}
class Teacher {
    string name;
    public:
        Teacher (string name): name{name}{}
        string getName() {return Name;}
};

class Department {
    Teacher *loT[100];
    int tNum;

    public: 
        Department(): tNum{0}{}
        void addTeacher(Teacher *teacher) {
            loT[tNum] = teacher;
            ++tNum;
        }
};

int main() {
    Teacher *teacher1 = new Teacher("Bob");
    Teacher *teacher2 = new Teacher("Jerry");
    {
        Department d;
        d.addTeacher(teacher1);
        d.addTeacher(teacher2);
        //d does not exist after this line/we exit scope!
    }
    cout << teacher1.getName() << "and " << teacher2.getName() << " are alive."<< endl; //This is valid.
    //If we end here, we would get a memory leak!  A does not destroy B, Department does not destroy Teachers.
    //If you're confused, note that Department d does not use heap, so we can simply let scope do its work.
    delete teacher1;
    delete teacher2;
}
\end{lstlisting}
\item If we destroy Department, then the Teacher still exists.  Note the default destructor is a shallow destructor --- it WILL NOT DESTROY TEACHER INSTANCES.
\item Same with the default copy ctor --- it is a shallow copy!
\item In UML, we represent this with [Department]<>------[Teacher]
\end{itemize}

\section{Inheritance}
\begin{itemize}
\item This is basically a class which is the parent class with changes.  This is also known as specialization/generalization.
\item Specialization is inheriting down.  That is, a Dog is a more specialized Animal.  It is more specific.
\item Generalization is going up.  An Animal is a very general way to describe a Dog. 
\item Basically, all Dogs are Animals, but not all Animals are Dogs.
\item We use a simple arrow pointing from the specialized children/subtype to the parent/supertype.
\item Why do we need inheritance?
\item Well, if some classes are very similar in fields to another class, where there are only some changes and are nearly identical, we can use a parent-child relationship to resolve this.
\item We can also use a union. 
\item Union is similar to struct.  The size of union is the size of its largest field, and it holds at most one of its field.  Unions do not have a destructor.
\item For example:
\begin{lstlisting}
union BookTypes{ Book *b, Text *t, Comic *c };
BookTypes myBook[20]; //Every item is a pointer to book, text, or comic, but each item is not more than one.
\end{lstlisting}
\item We can make a superclass by using the ``union'' keyword in C++:
\begin{lstlisting}
union book {
    Text t;
    Comic c;
}
\end{lstlisting}
\item But there's a better way to do inheritance in C++, without union or void pointers.
\item As a refresher from Java:
\begin{itemize}
\item Composition is ``owns a''.  For example, a Room class depends on a House class to exist.  Remove the House, and there isn't any Rooms.
\item Aggregation is ``has a''.   For example, a Children class can exist without a Class class.
\item Inheritence is ``is a''.  For example, a Dog has the features of a Animal class.
\end{itemize}
\begin{lstlisting}
class Parent {
    public:
        int id_p;
};

class Child : public Parent {
    public:
        int id_c;
        //id_p is inherited.
};
\end{lstlisting}
\item Another example from class:
\begin{lstlisting}
class Book {
    //string title, author;
    //int numPages;
    //If we do not make them protected over protected, then we will not be able to access the fields with children

    protected string title, author;
    protected int numPages;
    
    public:
        Book(title, author, numPages) {}
};

class Text : public Book {
    string topic;

    public:
        Text(title, author, numPages, topic): 
            Book{title,author,numPages}, //USE MIL
            topic{topic}
        {}
};

class Comic: public Book {
    string hero;

    public:
        Comic() {}
};
\end{lstlisting}
\item Note that when the object is costructed, now:
\begin{itemize}
\item Space is allocated.
\item The superclass fields are constructed.
\item Fields are constructed.
\item The ctor body runs.
\end{itemize}
\item Thus, when we initalize a subclass, it constructs the superclass fields AND THEN constructs its additional fields.
\item Every subclass inherits everything from the superclass, including fields and methods, if they are public or protected!  If they are private, it will not be accessible.
\item Thus, we have two problems if we didn't do the above: we wouldn't be able to access private fields/members, and we need the superclass to be constructed before the subclass.
\item If the superclass has no default ctor, the subclass must have a non-default superclass ctor call in the MIL.
\item Public means that anyone can access field.
\item Private means only the class itself can access the field.
\item Recall protected means we can access it directly within the class or through inheritance.  For example, a protected field in Book can be accessed by Comic and Text, but a protected field in Text cannot be accessed by Comic or Book.
\item What happens to private or protected fields that we inherit from a parent?
\item If we try:
\begin{lstlisting}
class Parent {
    public:
        int a;
    private: 
        int b;
    protected:
        int c;
};

class Child : public Parent {
    //Can access a publicly, c protected, b is private and inaccessable
};
class Child : protected Parent {
    //Can access a and c protected, b is private and inaccessable.
};
class Child : private Parent {
    //Cannot access anything; all private.
};
\end{lstlisting}
\item We can use accessors and setters, and make them protected, to only allow subclasses to access them, but still protect private fields.
\item ``Book b = Comic {...}'' will create a Book object.
\item When we initalize a superclasswith a subclass, we ``slice'' --- we cut off the extra fields that are not used by the superclass. 
\item Consider the following:
\begin{lstlisting}
Comic c {" ", " ", 40, " "};
Book *pb = &c;
Comic *pc = &c;
pc -> isItHeavy(); 
pb -> isItHeavy();
\end{lstlisting}
\item When we access an object through pointers, there is no slicing (no need and will not occur).  But pc-> will return the Comic value, and pb-> will return the Book value.
\item The overriding depends on the type of the pointer, not what the pointer is pointing at!
\item Our problem is that the same object behaves differently, depending on the type of the pointer that points at it, even though slicing does not happen.
\begin{lstlisting}
class Book {
    //...
    virtual bool isItHeavy() const {...} //add virtual
};

class Text: public Book {
    //...
    bool isItHeavy() const override {...} //add override
}
\end{lstlisting}
\item This will force the earlier example work logically --- that is, a Book pointing to a Comic will still run a Comic's override.
\item Remember to use the virtual and override keywords when overriding functions!
\item This thus fixes our earlier problem.
\item The above fix will resolve the issues of:
\begin{lstlisting}
Comic c {" ", " ", 40, " "};
Book *pb {&c};
Book &rb{c};
Comic *pc{&c};
pc -> isItHeavy(); //true
b.isItHeavy(); //true
pb->isItHeavy(); //true
\end{lstlisting}
\item Note if you have a method in a subclass, you can't invoke it from a superclass.
\item Now, what we can do is make an array of Book pointers, and by using virtual, we can access the correct methods.  And remember that pointers do not slice.  Thus, an array of Book pointers can easily handle all Books and its subclassses.  Or IOW, an array of the parent class will work as an array of it and its subclasses.
\item This trait --- being able to deal with multiple forms of a superclass and its children --- is known as polymorphism.  That is, we can abstract to accomodate everything.
\item For example, istream is a superclass for ifstream, istringstream, etc.  Based on the parameter (which is a reference), it will call the appropriate override.
\item See C++/inheritance/*
\item If we want to use polymorphism, do NOT use array of objects.  Always use an array of pointers to said object(s).
\item It is a good idea to always make the destructor of a superclass virtual, and override the dtor of subclasses, even if the dtor doesn't do anything.
\item Even if we want to use the default dtor, we should instead create an empty dtor body and add the virtual keyword.
\item Add the ``final'' keyword to a class declaration to specify that this subclass will not have children:
\begin{lstlisting}
class y final: public X {
};
\end{lstlisting}
\item What if we want to forbid the user from creating a superclass?  For example, maybe the superclass is always either one subclass or another --- and we don't want the user to be able to make said superclass.
\item For example:
\begin{lstlisting}
class Student {
    protected:
        int numCourses;
    public:
        //We state that this function has no implementation in this class.
        //This is called a pure virtual method.  It allows the compiler
        //to not complain about a lack of a definition.
        //A class with a pure virtual method cannot be instantiated.
        //This is called an abstract class (think Java).
        virtual int fees() const = 0;
};

class Regular: public Student {
    public:
        int fees() const override;
};

class Coop: public Student {
    public:
        int fees() const override;
};
\end{lstlisting}
\item As mentioned, a class with a pure virtual method is an abstract class.
\item Its purpose is to organize the subclasses.  For example, we group Regular and Coop students together as ``Students'', but we don't want the user to call the Student superclass.
\item Subclasses are also abstract unless they implement all the pure virtual methodds.
\item We call classes that can be instantiated (that is, they have no pure virtual methods) as ``concrete'' classes.
\item In UML, all abstract classes are in italic, and all virtual/pure virtual methods are in italics.
\item See c++/purevirt
\end{itemize}

\section{Inheritance --- Copy/Move Operators}
\begin{itemize}
\item See c++/inheritance
\item What happens if we make the subclass with no copy/move ctors/assign, but the superclass has them?
\item For example:
\begin{lstlisting}
class Book {
    public:
    //Assume copy/move operations are defined.
};

class Text : public Book {
    string topic;
    public:

        //copy ctor
        Text::Text(const Text &other) :
            Book{other}, topic{other.topic}
            //We slice other to fit Book.
        {}

        //copy assignment
        Text& Text::operator=(const Text &other) {
            Book::operator=(other);
            topic = other.topic;
            return *this;
        }

        //move ctor
        Text::Text(Text &&other) :
            Book{std::move(other)}, topic{std::move(other.topic)}
        {}

        //move assignment
        Text & Text::operator=(Text &&other) {
            Book::operator=(other);
            topic = other.topic;
            return *this;
        }
};
\end{lstlisting}
\item Let's say we called:
\begin{lstlisting}
Text t1{...}, t2{...};
Book *pb1 = &t1, *pb2 = &t2;
*pb1 = *pb2;
//This will call the Book CAssOp.  We don't have a virtual!
//We call this a partial assignment - the compiler sees this as a Book due
//to pointers, so it only partially assigns the value.
\end{lstlisting}
\item But we realize that this doesn't work for the operator overrides --- we return *this, but *this is not the same in the Book and Text operators!  
\item Luckily, C++ is smart, and deals with this for us, so we can continue to use virtual/override as per usual.
\item Consider:
\begin{lstlisting}
Text t{...};
Book b{...};
Text *pt = &t;
Book *pb = &b;
*pt = *pb; //This invokes the subclass now, not the superclass.
\end{lstlisting}
\item Note that you can't override a virtual function with different parameters.  We don't want to mix sub/superclasses.
\item This is why we have problems of partial and mixed assignment.
\item We solve by making an abstract superclass, instead of using, say, a Book -> subclass hierarchy, use a AbstractBook -> Book and subclass hierarchy.
\item We don't write any of the Big 5 in the abstract superclass --- we avoid the override problem by straight up removing the Big 5!
\item Basically, always make your superclass abstract.  Easiest way is to make the dtor a pure virtual method.
\item Don't forget to impelment a destructor if we do this, even if we don't need to implement a dtor that does deep copy destroy, as it has to be implemented to not be abstract.
\item To avoid assigning one subclass to another, add the protected keyword to the abstract assignment override.  
\end{itemize}
\end{document}
