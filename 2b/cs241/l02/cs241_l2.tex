\documentclass[12pt]{article}

\usepackage{times}
\usepackage{textcomp}
\usepackage{listings}

\author{Clement Tsang}

\begin{document}

\begin{center}
\Large\textbf{CS 241, Lecture 2 - Characters and Machine Language}
\end{center}

\section{Characters}
\begin{itemize}
    \item ASCII:
        \begin{itemize}
            \item 0-31: control
            \item 48-57: 0-9
            \item 65-90: A-Z
            \item 97-122: a-z
            \item $|$A-a$| =$ 32
        \end{itemize} 
\end{itemize}

\section{Bitwise Operations}
\begin{itemize}
    \item In C, we have bitwise operations.
    \item \lstinline{~} (not) flips all bits.
    \item \lstinline{&} (and) will give the AND of two binary strings.
    \item \lstinline{|} (or) will give the OR of two binary strings.
    \item \lstinline{^} (xor) will give the XOR of two binary strings.
    \item \lstinline{>>} and \lstinline{<<} (right and left shift, respectively) will do a logical shift right and left by the given integer, respectively.  A left shift is equal to multiplying by 2.
        \begin{itemize}
            \item Logical shift: Always shift zeroes for both left and right.
            \item Arithmetic shift: If left, shift zero.  If right, shift MSB.
            \item We get undefined behaviour if we right shift a signed integer.  
            \item For our case, we will ALWAYS use logical shifts!
        \end{itemize} 
\end{itemize}

\section{Instructions}
\begin{itemize}
    \item Programs reside in the same place on the data they operate on - they are also data (von Neumann architecture).
    \item Thus, we cannot really distinguish, without more information, whether a data is a program (set of instructions) or if it just data!
    \item For this course, we will work with 32-bit MIPS.
    \item Our MIPS CPU contains 32 registers, \$0, ... ,\$31, and hi, lo.
    \item Our control unit contains our PC and instruction register (IR).
    \item We have two memory registers - memory data register (MDR) and memory address register (MAR).
    \item We also have an ALU.
    \item We use a bus to transfer data from CPU to memory.
    \item Wew have a few types of memory that we use in computers (from fastest to slowest):
        \begin{itemize}
            \item CPU/registers
            \item L1 cache (SRAM)
            \item L2 cache (SRAM)
            \item RAM (DRAM)
            \item disk
            \item network memory
        \end{itemize}
    \item Each register can be represented in 5 bits.
    \item The registers we get to control with MIPS are our 32 registers (and hi and lo).  Note some registers are special and we do not need to touch:
        \begin{itemize}
            \item \$0 is ALWAYS 0
            \item \$31 is used for return addresses
            \item \$30 is our stack pointer
            \item \$29 are frame pointers
        \end{itemize}
    \item Note that in most MIPS standards, \$29 and \$30's roles are reversed.  Why?  Who knows.
    \item MIPS takes instructions loaded from RAM and executes them.
    \item To solve our earlier problem with confusing data and programs, we will state that anything with memory address 0 in RAM is an instruction.
    \item After we execute an instruction, we go to the NEXT instruction labelled by our PC.
    \item We use a program called a loader to put our program into memory, and then sets the PC to be the first address.
    \item CPUs basically follow a fetch-execute cycle, in which the following occurs:
    \begin{lstlisting}[mathescape, numbers=left]
PC = 0
while true do
	IR = MEM[PC]
	PC += 4
	Decode and execute instruction in IR
end while
    \end{lstlisting}
    \item Eventually some instruction will break out of the infinite loop.
\end{itemize}

\section{Adding, lis, and returning}
\begin{itemize}
    \item Let us write an example program that takes in registers \$8 and \$9 and stores it in \$3:
        \begin{lstlisting}[mathescape, numbers=left, breaklines=true]
; add $\$$3 $\$$8 $\$$9 is what we want - binary equiv. is 00011, 01000, 01001, respectively.
; we see that our binary equivalent is 0000 00ss ssst tttt dddd d000 0010 0000
; now let us insert our value, and get 0000 0001 0000 1001 0001 1000 0010 0000
; so we thus reduce this to a 32-bit word in hexadecimal: 0x01091820
.word 0x01091820
.word 0x03e00008 ; jr $\$$31, return
        \end{lstlisting}
    \item Note we HAVE to jump to register \$31 to end our program - we do so with \lstinline[mathescape]{jr $\$$31}.
    \item But this assumes you had values \textbf{in} the register - how do we put immediate values in registers?  We use the \lstinline[mathescape]{lis $\$$d} MIPS command. 
    \item Note that \lstinline{lis} is \textbf{non-standard}, normally we would use \lstinline{addi} to add immediate values. 
    \item This will place the next word, immediately after, into register \$d.  It will increment the PC by 4 and skip the next line as it is (usually) NOT an instruction.
    \item Note this value is a two's complement integer.
    \item Let us write a MIPS program that adds 11 and 13:
    \begin{lstlisting}[mathescape, numbers=left]
.word 0x00000814 ; lis $\$$1
.word 0x0000000b ; ivalue 11
.word 0x00001014 ; lis $\$$2
.word 0x0000000d ; ivalue 13
.word 0x00221820 ; add $\$$1 and $\$$2, store in $\$$3
.word 0x03e00008 ; jr $\$$31
    \end{lstlisting}
\end{itemize}

\section{Multiplication and Division}
\begin{itemize}
    \item We get a problem with multiplying and division - we may need more space when multiplying (ie: $2^{30} \times 2^{30} = 2^{60}$), and when dividing, we want the quotient and remainder!
    \item For multiplication, we COMBINE the hi and lo registers to get a 64-bit register.  The most significant word is placed in hi, and least significant word in lo (most sig word is largest 4 bytes).
    \item For division, we put the quotient \lstinline[mathescape]{$\$$s $\div$ $\$$t} in lo and the remainder \lstinline[mathescape]{$\$$s % $\$$t} in hi.
    \item To access data from hi and lo, we use the \lstinline[mathescape]{mfhi $\$$d} and \lstinline[mathescape]{mflo $\$$d} instructions to move from the hi/lo register to the register we state.
\end{itemize}

\end{document}

