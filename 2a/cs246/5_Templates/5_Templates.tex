\documentclass{article}
\usepackage{times}
\usepackage{textcomp}
\usepackage{listings}

\author{Clement Tsang}

\begin{document}

\section{Templates}
\begin{itemize}
\item We can use templates to, say, create a list that can handle both chars and ints, instead of making two lists each of one primitive type.
\item A template is a way to make a generic function/class --- it fits all types.
\item We can create and call a template class:
\begin{lstlisting}
template <typename T> class List {
    //...
};

int main() {
    List<int> lista;
    List<char> listb;
    List<List<int>> listc;
}
\end{lstlisting}
\end{itemize}


\section {Standard Template Library (STL)}
\begin{itemize}
\item Vectors (dynamic length arrays):
\begin{itemize}
\item #include <vector>
\item Basically the better way of making dynamic arrays.  Safer, avoids you having to new/delete, etc.
\item Is generic; can handle any type.  Unlike arraylists in Java, it can handle primitives.
\item As such, when we declare, we must say, for example, ``vector <int> v{4, 5};''.
\item The ``emplace\_back(value)'' method adds a value to the back of the vector that calls it,
\item Vectors also have iterators:
\begin{lstlisting}
for (vector<int>::iterator it = v.begin(); it != v.end; ++it) {
    cout << *it << emd;
}

//equivalent to:

for (int i = 0; i < v.size(); ++i) {
    cout << v[i] << endl;
}

//We can replace vector<int>::iterator with just the auto keyword.
//Now don't use it too much.  It gets confusing.

//If we want to go in reverse:
for (auto it = v.rbegin(); it !=  v.rend(); ++i) {
    //...
}
\end{lstlisting}
\item Use ``v.pop\_back();'' to remove the last item.
\item ``v.erase(v.begin() + 3);'' will remove the 4th item.
\item Assume we have an iterator. We could do v.erase(it).
\item What if we want to erase the LAST element?  It's v.erase(v.end() - 1), NOT just v.end().  This is because the .end() returns an iterator to the last value of the list, but that, unfortunately, is a nullptr, and NOT the actual last element of the list.
\item Note that .begin() differs from .front(). 
\item The difference between v[i] and v.at(i) is that the former will just error and crash out, but the latter throws an exception.  We can use this to try/catch and gracefully stop the program, or work around it.
\end{itemize}

\item Try and Catch
\begin{itemize}
\item #import <stdexcept>
\item If try succeeds, catch will be ignored.
\item Every catch is seperate.  We use multiple catches for multiple specific errors.
\item Consider this example:
\begin{lstlisting}
void f() {
    throw out\_of\_range("oops");
}

void g() {
    f();
}

void h() {
    g();
}

int main() {
    try {
        h();
    }
    catch (out\_of\_range r) {
        //This will catch h() if it throws out of range!
    }
}
\end{lstlisting}
\item If you set the parameter of the catch to (...), then it will catch any thrown exception.
\item See c++/exception/
\item we note that exceptions are not fast at all --- we should never put, say, all our code in a giant try/catch block.  it should only be used on sensitive statements that have a high probability of hitting an exception.  
\item We can define our own exceptions.  For example:
\begin{lstlisting}
class BadInput : public runtime\_error {
    BadInput(const string& msg):
        runtime\_error(msg+ "how did you manage to screw up")
    {}
};
\end{lstlisting}
\item This will allow you to create your own exception that the original exception class does, but you can modify the message.
\item Always try to throw by value, and catch by reference.  
\end{itemize}
\end{itemize}

\section{Design Patterns}
\begin{itemize}
\item For certain problems, we have certain solutions.  
\item Thus, if a solution exists already, why bother coming up with a new solution --- just adapt known stuff to fit our needs!
\item Abstract solutions are useful here for that reason.
\item Observer Pattern:
\begin{itemize}
\item Follows the publish-subscribe model.
\item Behavioural pattern.
\item One class: publishers/subjects generate data.
\item Some subscribers/observers recieve data and react to it.
\item Sequence of calls:
\begin{itemize}
\item Subject's state is updated.
\item Subject notifies the observer(s) (notifyObservers) that there is a change in the data through the observer's notify().
\item Each observer calls concreteSubject::getState() to query the state and reacts on it,
\end{itemize}
\end{itemize}
\item Decorator Pattern:
\begin{itemize}
\item The general idea is to use this pattern whenever we want to add behaviour during run-time, rather than in advance.
\item We can do so by subclassing the Component class into a Decorator class (with the additional component), then making our original ConcreteComponent a subclass of the Decorator.
\end{itemize}
\item Factory Method Pattern:
\begin{itemize}
\item We have two main abstract superclasses: Products and Creators.  Each of course has concrete subclasses.
\item Our Creator has a virtual factory method that creates an object of type Product.
\item Basically, instead of Product being in charge of creating, Product just calls Creator and Creator figures out what to do.
\end{itemize}
\item Template Method Pattern:
\begin{itemize}
\item We declare some functions in a general superclass.
\item If we have a general algorithim, but we want to redefine parts of that algorithim, we can make the parts of the algorithim private, and let subclasses override said virtual methods.
\item Make algorithim functions private to force user to do the entire algorithim, rather than just parts of it.
\end{itemize}
\item This technique --- using a private pure virtual method, and making wrapper functions to allow the client to interact with, protecting the client.
\item NVI idiom: Non-Virtual Interface:
\begin{itemize}
\item All public methods should be nonvirtual.
\item All virtual methods should be private, except the destructor (if you choose to do so).  This should be for very obvious reasons.
\end{itemize}
\item STL: map
\item For example:
\begin{lstlisting}
#include <map>;
map<string, int> m;
m["abc"] = 1;
m["def"] = 4;
cout << m["abc"] << endl;
//If hte value exists, it will output the mapped value.
cout << m["ghi"] << endl;
//If the value does not exists, it will be added mapped to a default value of 0.
m.erase("abc");
if (m.count("def")) cout << "found" << endl; 
//count returns 1 if the key exists.

for (auto &p : m ) {
    cout << p.first << ' ' << p.second << endl;
}
//map supports interators.
//note that first and second are fields, NOT methods.
//Note that iterating goes through by the SORTED order.
//The above code would output:

//1
//0
//found
//def 4
//ghi 0
\end{lstlisting}
\item Visitor Pattern:
\begin{itemize}
\item We split into Element and Visitor.
\item During runtime, the Visitor will ``visit'' the Element.
\item The visitor should have a virtual function that is overloaded based on a pointer to the element subclasses.  It should be overloaded.
\item The element should have a virtual function that accepts.
\item Basically, the element will have a function that accepts (takes in) a visitor.  This visitor is in charge of the behaviour of the element.  The element itself does not include this behaviour built in.
\item Good for modifying existing classes to enable further functionality.
\end{itemize}

\item Circular Include Dependencies:
\begin{itemize}
\item Let's say we simply had a header that had a field that is a pointer to another class.
\item We don't actually need to know the space of this variable.  
\item We can use forward declaration to allow us to not need to include the header of another class.
\item This is very helpful if we have a class that relies on a class that relies on that same class.
\item The problem is that if we want to initalize that value, we must actually know how the class is implemented.  Then, we will need to include.
\item If we need to include, then put the header includes in the .cc file if possible.  You want to always minimize the number of .h includes in your own header file.
\end{itemize}

\end{itemize}

\section{Idioms}
\begin{itemize}
\item PIMPL (pointer to implementation) idiom:
\begin{itemize}
\item We don't want our users accessing our fields.
\item We put all our private fields that we do not want them to access into a struct.
\item We then create a pointer to that struct as our private field for the user to access in our main class.
\item In our main class, we forward declare the impl. class.
\item Our .cc will have to include the header to the struct in order to access said fields for functions.
\item Good form of implementation hiding.
\item This means that if we update the implementation of window.h and window.cc, we only have to recompile XWindow, not XWindowImpl.
\end{itemize}
\end{itemize}

\section{Coupling and Cohesion}
\begin{itemize}
\item As in CS136, we want to have low coupling and high cohesion.
\item Coupling is when models depend on each other.  Officially, it's the degree to which the program models depend on each other.
\item We can obtain low coupling by using less includes and more forward declaration.
\item Friends will increase coupling, so minimizing their use will lower our overall coupling.
\item Using global data increases coupling.
\item Cohesion is how closely elements of a specific module are related to each other.
\item We can easily achieve high cohesion by making all modules as specific as possible.  For example, a math operations module should only do math operations.  A graphical module should only do graphics.
\item Many of our patterns will encourage cohesion by encouraging us to split up modules into specific parts.
\item The Single Responsibilitiy Principle states that a class should have responsibility over a single part of the functionality in the software, and that the responsibility should be encapsulated.
\item We can do this by seperating classes, for example, seperating logic and graphics.
\end{itemize}

\section{Model View Controller --- MVC} 
\begin{itemize}
\item The goal of this model is to seperate the distinct notion of the data (``model''), the presentation of the data (``view''), and the manipulation of data (``controller'').
\item Essentially: User -> Controller -> Model -> View -[sees]-> User
\end{itemize}

\section{Exception Safety --- RAII}
\begin{itemize}
\item There is a problem with exceptions --- if we throw  one, we might leak heap memory if we had to delete it assuming there was no throw.
\item For example, if there is a delete after the throw, but the throw is called, the delete is never run, and we have a memleak.
\item You would think that the compiler would delete pointers if an exception is raised, but unfortunately, when an exception is raised, the following things will occur:
\begin{itemize}
\item Stack unwinding:  All STACK allocated data is cleaned up.  Destructors are all run, and memory is reclaimed.
\end{itemize}
\item But the problem is that all HEAP allocated data is not cleaned up!  Thus, something like:
\begin{lstlisting}
void f() {
    MyClass *p = New MyClass;
    MyClass mc;
    g(); //throws exception
    delete p;
}
\end{lstlisting}
\item p might be leaked.
\item A solution would be to put whatever may reach an exception within a try-catch block.
\item But a better solution is to take advantage of stack unwinding to stop any memory leaks.
\item We use RAII (Resource Acquisition Is Initialization).  
\item Every resource sohuld be wrapped in a stack-allocated object, whose destructor will delete it.
\item By doing so, NOW we can store data on heap, but if we reach an exception, by stack unwinding, the data will be deleted through the destructor of the stack-allocated item.
\item This is has the added benefit of making resource memory incredibly easy.  As we will make as many resources wrapped in a stack-allocated object, we will avoid having to call (and possibly forget) delete calls to free memory.
\item fstream already follows this idiom to make sure files are closed properly.
\item We can use unique_ptr<T> class that holds a pointer of type T, which provides a dtor that will delete said pointer.  You can treat this like a regular pointer, so you can dereference.  Include <memory>.
\item We use ``auto NAME = std::make_unique<CLASS>(VALUE);''
\item What happens with the following?
\begin{itemize}
class c{...}
unique_ptr<c> p {new c{...}};
unique_ptr<c> q = p; //copy ctor invoked, buuuuut it gives an error.
\end{itemize}
\item The copy ctor will give an error.  It's in the name --- it's a unique pointer!  If copy ctor did work, it would result in a double-free upon the destruction of the stack.
\item In other words, since it does not deep copy, deleting one will delete the same pointer that p relies on.
\item Thus, the copy ctor for unique_ptr is disabled.
\item The unique_ptr class uses ``= delete'' on its copy ctor/assignment to ensure there is no copying.
\item Note that the MOVE operation is still fine --- this is because move is for an rvalue, and we will never be able to ``move'' an on-stack unique_ptr.  We just cannot COPY an existing unique_ptr!
\item What if, for some reason, we need to have more than one pointer pointing at the same data on heap?  That is, we want more than one unique_ptr pointing at the same thing on the heap.
\item Simple!  Make one of your pointers a unique_ptr, and all other ones just normal pointers!  Make sure your unique_ptr is popped last (so declare it first), so when stack is destroyed/unwinded, we will never have a double-free.
\item Note the above only works if you do not call delete on any of the normal pointers!  
\item But what if, for some god-forsaken reason, we want to have many pointers pointing to the same location on heap, but we want each pointer to OWN the data, like unique_ptr?  And we don't want to have to deal with the bullshit of making sure the unique_ptr is popped last?  We have something for that, called std::shared_ptr.
\item This thing even COUNTS how many pointers are pointing at the same memory address.  When we delete a pointer, the counter goes down, and only if the counter hits zero will the memory be freed.  This is called a reference count.
\item Because of this, copying is valid with shared_ptrs.
\item We use ``auto NAME = std::make_shared<CLASS>(VALUE);''
\item In this way, we can have multiple pointers that own the heap memory, as well as the order of how we pop not mattering!
\item To sum up, we used the shared and uniqute pointers intsead of raw pointers as much as possible!  This reduces the chance of memleaks.  For more complicated resources, you'll have to create your own wrapper class.
\item We have three types of exception safety for a function f (from weakest to strongest):
\begin{itemize}
\item Basic guarantee: If the function throws an exception, the program is still valid.  No memory is leaked.  State of variables may have changed.  Invariants are preserved.
\item Strong guarantee:  If f throws an exception, when the handling of the exception is completed, it will be as if function f is never run.
\item No-throw guarantee:  The function will never throw an exception.  It is ``perfect'', but the hardest to do.  Basically, we will be in charge of catching ANY case that would throw an exception, and ensure the function f itself does not throw it.  It will always accomplish its task no matter what.  Basically what you would use formal verification for.
\end{itemize}
\item In other words, we sometimes want to avoid the possibilty of any exceptions being thrown.  No-throw would be the best.
\item An example:
\begin{lstlisting}
class A {
    ...
    public:
        g(); //strong
    ...
};

class B {
    ...
    public:
        h(); //strong
    ...
};

class C {
    A a;
    B b;
    public:
        void f() {
            a.g();
            b.h();
        }
};
\end{lstlisting}
\item A doesn't have handler code, so we know it is not a ``no throw''.
\item To make f() strong, we have to check where it fails, and undo the actions done.
\item If a.g() fails, then it is simple, just stop right there.  As g() is strong, it will undo any changes.
\item If b.h() fails, then we have to also undo whatever a.g() has done.  The changes by h() are simple, as the function itself is strong.
\item But what if we can't really undo the change?  For example, maybe the data we changed has already been further processed.  If this is the case, then a strong guarantee may not be possible.
\item In other words, from our example, we can't actually really guarantee strong or even basic.  If global data is changed, or data that is modified is already being used, undoing may be impossible.
\item If g() and h() don't have any non-local side effects, then we can use the ``copy and swap'' trick:
\begin{lstlisting}
void c::f() {
    A atmp = a;
    B btmp = b;
    atmp.g(); //If an exception is thrown, nothing happens to the original!
    btmp.h();

    // No exceptions raised.
    a = atmp;
    b = btmp; //but what if copy raises an exception?
}
\end{lstlisting}
\item We note that you cannot fail in copying a POINTER.  Thus, we take advantage of this,
\item PIMPL idiom can work here.  For example:
\begin{lstlisting}
struct CImpl P {
    A a;
    B b;
};

class C {
        std::unique_ptr<CImpl>pImpl;
        public:
            void f() {
                auto tmp - make_unique<CImpl><*PImpl);
                temp->a::g();
                temp->b::h();
                std::swap(pImpl, tmp);
}
\end{lstlisting}
\end{itemize}

\Section{STL vector and exception safety}
\item Vectors use heap-allocated memory, but also use RAII to ensure the vector itself is not leaked.
\item However, it does not guarantee what will happen to its contents.
\item To avoid dealing with delete and possible exeception-caused leaked memory, if we, say, have a vector of pointers, it is smarter to use a vector of unique_ptrs,allowing us to not need explicit deallocation.
\item We can also use ``emplace_back'' versus ``push_back'' when we are adding elements to our vector.  Using emplace_back gives our vector a strong guarantee,
\item But what does emplace do?
\begin{itemize}
\item If we need to resize, then we allocate a new, larger array.
\item Use a copy ctor to copy objects.
\item If the copy ctor throws, we catch and throw away the new array, retaining the original array (strong).
\item Otherwise, we copy completely to new array and delete the old one.
\item We will use the move ctor INSTEAD of the copy ctor if the compiler can guarantee that the move ctor is no-throw.  This is because otherwise, if move ctor throws an exception, move will affect the original array, and thus result in emplace not being strong.
\item We can mark the move ctor as having no exceptions with the ``noexcept'' keyword to show it is a no-throw ctor.
\item For example:
\begin{lstlisting}
class C {
    public:
        C(C && other) noexcept {...}
        C &operator=(C && other) noexcept {...}
        ...
};
\end{lstlisting}
\item As a guideline, if we know that a method will never throw or propagate an exception, mark it as noexcept.  Moves and swaps should be noexcept.
\end{itemize}

\Section{Casting}
\begin{itemize}
\item Do NOT use C style casting.
\item We have four different ways of casting in C++:
\begin{itemize}
\item static_cast: we use this for sensible casts, where there is well defined behaviour.  For example, double to int.  We can do ``static_cast<int>(d)'', where d is a double.
\item reinterpret_cast: for unsafe casts that are implementation dependent.  For example, casting a Student to a Turtle: reinterpret_cast<Turtle*>(&s).  Who knows what'll happen?
\item const_cast: used to add or remove ``constness' from something.  However, beware of ``const poisoning'' which may occur due to this, as you might set off a chain reaction of issues by using const casting.  We use const_cast<int*>(p), where p was a const pointer to an integer.  
\item Note that we would use const casting to removed the const from something that doesn't need to be const - for example, do not try removing the const of a const int.  You can, however, remove the const of a constant integer pointer.
\item dynamic_cast: used to convert a subclass pointer to a superclass pointer.  We could use static_cast if we are sure that the conversion will work, but otherwise this is better.  If the conversion fails, it will be set to a nullptr.  We would do, for example, Text *pt = dynamic_cast<Text*>(pb);, where pb is a pointer to a Book.  Dynamic methods work only on classes w/ at least one virtual method, but that's kinda the same as saying it has to be done on a subclass.  
\item We use this over static_cast in situations where we are worried about it failing, as dynamic_cast will fail gracefully.
\item Note that all these casting types will work on smart pointers.
\item Of course, casting also works on references.  For example:
\begin{lstlisting}
Test t{...};
Book &book = t;
Text &tCast = dynamic_cast<Text &> book;
\end{lstlisting}
\item We have to note that if you try passing a reference to a dynamic cast, well, we can't assign ``nullptr'' to a reference!  In this case, if book didn't point at a text reference, this would actually raise an exception.
\end{itemize}
\end{itemize}

======END OF MATERIAL=======

\section{Exam}
\begin{itemize}
\item No BASH.  No Makefile.  Just C++, though this is all of C++.  Focus on the second half, but there's probably still big 5.
\item Iterators will definitely be on it (speculation).

\end{itemize}


\end{document}
