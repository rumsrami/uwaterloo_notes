\documentclass{article}
\usepackage{times}
\usepackage{textcomp}
\usepackage{listings}

\author{Clement Tsang}

\begin{document}

\section{Templates}
\begin{itemize}
\item The compiler is smart enough that sometimes, we don't have to specify the type of the templated function.
\begin{lstlisting}
template <typename T>T min (T x, T y) {
    return x < y ? x : y;
}

int f() {
    int x = 1, y = 2;
    int z = min(x, y); //This will work w/o stating it's an int.
    int a = min<int>(x, y); //This will also work.
    char w = min('a', 'c'); //This works, T is a char,
    auto f = min(3.0, 5.0); //This works, where T and f will be a double. 
}
\end{lstlisting}
\item Note this only works for function templates.  NOT CLASSES.

\item Consider the example:
\begin{lstlisting}
void foreach(AbstractIterator start, AbstractIterator finish, int(*f)(int)) {
    while (start != finish) {
        f(*start);
        ++start;
    }
}

template<typename Iter, typename Func>
void for_each(Iter start, Iter finish, Func f) {
    while (start != finish) {
        f(*start);
        ++start;
    }
}

void f(int n) {cout << n << endl;}
int a[] = {1, 2, 3, 4, 5};
for_each (a, a+5, f); //Valid.
\end{lstlisting}
\item The first version will possibly fail in a few situations.  If the AbstractIterator doesn't support ++, !=, or *(dereferencing), then it will not work.  If the function we pass isn't callable (no idea how the hell that would happen), it will also not work.
\item We look at the second example.  That's valid!  We don't have to explicitly claim the type of each typename, as above, due to this being a function template.
\item We could change all the names of Iter and Func and it would still work.
\end{itemize}


\section{Algorithms}
\begin{itemize}
\item From the <algorithm> library.
\item It contains a suite of template functions, many of which work over iterators.
\item We can use these on the final/assignment, but not needed at all.
\begin{lstlisting}
for_each() {
    //seen above.
}

template<typename Iter, typename T>
Iter find(Iter first, Iter last, const T &val) {
    //returns iterator to first element in first (inclusive) to last (exclusive)
    //if val is not found, returns last.
}

int count(Iter first, Iter last, const T &val) {
    //returns the number of occurances of val.
}

template<typename InIter, typename OutIter>
OutIter copy (InIter first, InIter last, OutIter result) {
    // copies the values from first (inc) to last (exc)
    // starting from result.
}

OutIter transform(InIter first, InIter last, OutIter result, Func f) {
    //Basically for_each and copy combined.
}

\end{lstlisting}
\end{itemize}

\section{Lambda}
\begin{itemize}
\item Recall we really only used lambda so we could avoid defining functions that we only really used once, so we didn't need a name.
\item We can do that in C++ as well.  This is useful for template algorithms, as we may only need the required function once.
\item We can achieve this like so:
\begin{lstlisting}
vector <int> v {...};
bool even (int n) {return n % 2 == 0;}
int x = count_if(v.begin(), v.end(), even); //Obviously valid

//OR

int y = count_if(v.begin(), v.end(), [](int n){return n % 2 == 0;}); //Also valid!
\end{lstlisting}
\item Yes.  ``[]'' is our ``lambda''.  Note we don't have to specify the return type; assuming we aren't idiots, the compiler will be able to guess what our return type is.
\item If you want to SAVE this lambda, what you can do is say auto x = []{...};

\end{itemize}
<++>

\end{document}
